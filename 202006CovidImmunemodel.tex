
%\documentclass[preprint,authoryear,11.5pt,sort&compress,square]{elsarticle}
\documentclass[preprint,11pt,sort&compress,square]{elsarticle}
%\documentclass[preprint,11.5pt,authoryear]{elsarticle}
\usepackage{lineno}
\usepackage{amssymb}
\usepackage{amssymb,amsmath}
\usepackage{latexsym}
\usepackage{graphics,graphicx}
\textheight 230mm
\textwidth 165mm
\oddsidemargin=0mm
\headsep -1cm
\evensidemargin=0mm
%\baselineskip 38pt

%\textheight 230mm
%\textwidth 175mm
%\oddsidemargin=-7mm
%\topmargin=0mm
%\bottommargin=0mm
\renewcommand{\baselinestretch}{1.2}
\makeatletter
\newtheorem{theorem}{Theorem}[section]
\newtheorem{lemma}{Lemma}[section]
\newtheorem{Remark}{Remark}
\newtheorem{proposition}{Proposition}[section]
\newtheorem{Proof}{pf}
\usepackage{multirow}
\usepackage{color}
\usepackage{graphicx}
\usepackage{epstopdf}
\usepackage{amsmath}
\newcommand\diff{\,{\mathrm d}}
\numberwithin{equation}{section}
\usepackage{amssymb}
\usepackage{setspace}%使用间距宏包
\usepackage{colortbl}  % \extrarowheight用到
\journal{Journal of Differential Equations }
\renewcommand*\figurename{Fig.}
\usepackage{comment}

\bibliographystyle{abbrv}
\begin{document}
\linenumbers
\begin{frontmatter}

	\title{Modeling the theoretical framework of Covid-19 immunology}
	\begin{comment}
	\tnotetext[mytitlenote]{Research partially supported by the China Scholarship Council (PS), NSF grant DMS-1411476 (YL), NSFC grants  11631012 and 11571273 (YX), and   National Mega-project of Science Research  2017ZX10201101-002-002 (YX)}

	\author[XiaoSongaddress]{Shuangshuang Yin}



	\author[Louaddress2]{Yuan Lou}

	\ead{lou@math.ohio-state.edu}

	\author[Wuaddress]{Jianhong Wu}

	\ead{wu@math.ohio-state.edu}


	\author[XiaoSongaddress]{Yanni Xiao\corref{mycorrespondingauthor}}
	\cortext[mycorrespondingauthor]{Corresponding author}
	\ead{yxiao@mail.xjtu.edu.cn}

	\author[XiaoSongaddress]{Pengfei Song}
	\ead{song921012@stu.xjtu.edu.cn}

	\address[XiaoSongaddress]{School of Mathematics and Statistics, Xi'an Jiaotong University, Xi'an, ShaanXi, 710049, PRC}
	%\address[Louaddress1]{Institute for Mathematical Sciences, Renmin University of China, Beijing, 100872, PRC}
	\address[Louaddress2]{Department of Mathematics, Ohio State University, Columbus, OH 43210, USA}

	\address[Wuaddress]{Department of Mathematics, York University, Canada}
	\end{comment}
	\begin{abstract}

	\end{abstract}

	\begin{keyword}

		\MSC[2010] 35J55 \sep  35B32
	\end{keyword}

\end{frontmatter}

\linenumbers
\section{Within host dynamic model}
Elements considered:

1. Infection dynamics
\begin{itemize}
	\item Target cells:        suspected ($S$) and infected ($I$) cells;
	\item Virus:    $v$.
\end{itemize}



2. Innate immunity
\begin{itemize}
	\item Macrophage and DC cells:    $M$;
\end{itemize}

3. Humoral immunity
\begin{itemize}
	\item B cells: $B_{n},B_{a},B_{m}$ naive, activated, memory;
	\item Plasma cells: $P$
	\item Short-term and long-term antibodies: $A_{S}$, $A_{l}$.
\end{itemize}

4. Cellular immunity
\begin{itemize}
	\item CD8 T cells: $X_{n},X_{a},X_{m}$ naive, actived, memory;
	\item CD4 T cells: $Y_{n},Y_{a},Y_{m}$ naive, actived, memory;
	\item Th1 and Th2 cells: $T_{2},T_{1}$.
\end{itemize}

\subsection{Virus dynamics}
The following equation \eqref{virusdynamics} describes the dynamics of target cells ($s$),infected target cells($i$) and virus ($v$).


\noindent Target cells ($S$)
\begin{itemize}
	\item Growth: target cells grow with an imposed carrying capacity;
	\item Suspected ($S$) to infected($I$): Virions ($V$) bind to target cells ($S$) to infect them;
\end{itemize}

\noindent Infected cells ($I$)
\begin{itemize}
	\item Suspected ($S$) to infected($I$): Virions ($V$) bind to target cells ($S$) to infect them;
	\item Decay: death;
	\item Killed by binding to activated CD8 T cells($X_{a}$).
\end{itemize}

\noindent Virus (V)
\begin{itemize}
	\item infected cells ($i$) produce virions;
	\item Decay: death;
	\item Killed by M cells, short-term and long-term antibodies.
\end{itemize}

\begin{equation}\label{virusdynamics}
	\left\{
	\begin{aligned}
		% suspected infected target cells and virus
		 & \frac{{\rm d } S}{{\rm dt}}=\lambda_{s}-\beta S V-d_{S}S, \\
		 & \frac{{\rm d } I}{{\rm dt}}=\beta S V-k_{xi}X_{a}I-d_{I}I,                          \\
		 & \frac{{\rm d } V}{{\rm dt}}=\delta I-\beta SV-k_{asv}A_{s}V-k_{alv}A_{l}V-k_{mv}MV-d_{v}V.                          \\
	\end{aligned}
	\right.
\end{equation}



Immunity is mediated by the compartments resistant cells ($R$), M cells ($M$), antibodies ($A_{s}$(IgG,IgA) , $A_{l}$(IgM)) and activated CD8+ T cells ($x_{a}$), the dynamics of which will be described shortly.

\subsection{Innate immunity}



M cells($M$) is stimulated by infected cells ($T_{1}$), and decay at the rate $d_{m}$.

\begin{equation}\label{innateimmunity}
	%\left\{
	\begin{aligned}
		 & \frac{{\rm d } M}{{\rm dt}}=\lambda\left(1+\frac{k_{m}T_{1}}{T_{1}+K_{m}}\right)-d_{m}M.   \\
	\end{aligned}
	%\right.
\end{equation}

\subsection{Humoral adaptive immunity}

The humoral adaptive immune response is mediated by short time antibodies ($A_{s}$, IgG, IgA) and long time antibodies ($A_{l}$, IgM), which bind to virions and neutralise them, rendering them non-infectious. Naive B cells ($B_{n}$) are stimulated by virus and become activated B cells ($B_{a}$) to proliferate and differentiate into plasma cells ($P$) with stimulation of Th1 cells, which produce short and long antibodies ($A_{S}, A_{l}$,).  Activated B cells also differentiate into memory B cells. \eqref{humoralimmunity} describes these processes.
\begin{equation}\label{humoralimmunity}
	\left\{
	\begin{aligned}
		 & \frac{{\rm d } B_{n}}{{\rm dt}}=\lambda_{bn}-\frac{k_{bn}qMVT_{2}}{qMVT_{2}+K_{bn}}B_{n}-d_{bn}B_{n},\\
		 & \frac{{\rm d } B_{a}}{{\rm dt}}=\frac{k_{bn}qMVT_{2}}{qMVT_{2}+K_{bn}}B_{n}+\frac{k_{ba}qMV}{qMV+K_{ba}}B_{a}+\frac{q_{bm}k_{bm}qMV}{qMV+K_{bm}}B_{m}-\rho_{b}B_{a}-\rho_{bp}B_{a}-d_{ba}B_{a}, \\
		 & \frac{{\rm d } B_{m}}{{\rm dt}}=\rho_{b} B_{a}-\frac{k_{bm}qMV}{qMV+K_{bm}}B_{m}-d_{bm}B_{m},                                                                                                                       \\
		 & \frac{{\rm d } P}{{\rm dt}}=\rho_{bp}B_{a}-d_{p}P,\\
		 & \frac{{\rm d } A_{s}}{{\rm dt}}=\delta_{as}P-d_{as}A_{s},                                                                                                                                                         \\
		 & \frac{{\rm d } A_{l}}{{\rm dt}}=\delta_{al}P-d_{al}A_{l}.                                                                                                                                                         \\
	\end{aligned}
	\right.
\end{equation}


\subsection{Cellular adaptive immune response}
The cellular adaptive immune response is mediated by effector CD8 T cells ($X_{a}$) and effector CD4 T cells ($Y_{a}$). Virus stimulate the differentiation of effector  T cells ($X_{a}, Y_{a}$) from their naive counterparts ($X_{n}, Y_{n}$); effector CD8+ T cells then increase the death rate of infected cells. Some effector  T cells ($X_{a}, Y_{a}$) remain after a primary infection as memory  T cells ($X_{m},Y_{m}$). Effector CD4 T cells ($Y_{a}$) also differentiate to Th1 cells ($T_{1}$) and Th2 cells ($T_{2}$).
\eqref{cellularimmunity} describes these processes.


\begin{equation}\label{cellularimmunity}
	\left\{
	\begin{aligned}
		 & \frac{{\rm d } X_{n}}{{\rm dt}}=\lambda_{xn}-\frac{k_{xn}qMVY_{a}}{qMVY_{a}+K_{xn}}X_{n}-d_{xn}X_{n},                                                                                       \\
		 & \frac{{\rm d } X_{a}}{{\rm dt}}=\frac{k_{xn}qMVY_{a}}{qMVY_{a}+K_{xn}}X_{n}+\frac{k_{xa}qMV}{qMV+K_{xa}}X_{a}+\frac{q_{xm}k_{xm}qMV}{qMV+K_{xm}}M-\rho_{x}X_{a}-d_{xa}X_{a},                    \\
		 & \frac{{\rm d } X_{m}}{{\rm dt}}=\rho_{x} X_{a}-\frac{k_{xm}qMV}{qMV+K_{xm}}X_{m}-d_{xm}X_{m},                                                       \\
		 & \frac{{\rm d } Y_{n}}{{\rm dt}}=\lambda_{yn}-\frac{k_{yn}qMV}{qMV+K_{yn}}Y_{n}-d_{yn}Y_{n},                                                                                       \\
		 & \frac{{\rm d } Y_{a}}{{\rm dt}}=\frac{k_{yn}qMV}{qMV+K_{yn}}Y_{n}+\frac{k_{ya}qMV}{qMV+K_{ya}}Y_{a}+\frac{q_{ym}k_{ym}qMV}{qMV+K_{ym}}M-\rho_{yt1}Y_{a}-\rho_{yt2}Y_{a}-d_{ya}Y_{a}, \\
		 & \frac{{\rm d } Y_{m}}{{\rm dt}}=\rho_{y} Y_{a}-\frac{k_{ym}qMV}{qMV+K_{ym}}Y_{m}-d_{ym}Y_{m},                                                       \\
		 & \frac{{\rm d } T_{1}}{{\rm dt}}=\rho_{yt1}Y_{a}-d_{t1}T_{1},                                                                                                                \\
		 & \frac{{\rm d } T_{2}}{{\rm dt}}=\rho_{yt2}Y_{a}-d_{t2}T_{2}.                                                                                                                    \\
	\end{aligned}
	\right.
\end{equation}

\section{Basic reproduction number}

$$\mathcal{R}_{0}=\frac{\delta\beta\lambda_{s}}{(d_{S}+d_{v}d_{s})d_{I}}$$

\bigskip
\noindent{\bf{Acknowledgement.}}




\bibliography{references}


\end{document}

\endinput

